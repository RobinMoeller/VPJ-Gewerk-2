\newpage

\chapter{Ausblick}
Die Validierung hat gezeigt, dass ein zuverl�ssiges und funktionsf�higes, autonomes System mit der Potentialfeldmethode entstanden ist. Dennoch k�nnten noch ein paar Verbesserungen an dem Gesamtsystem vorgenommen werden. 
Eine Voraussetzung des vorliegenden Systems ist, das Hindernisse, wie Stuhlbeine, aus dem Konfigurationsraum entfernt werden, da diese nicht immer von den Infrarot-Sensoren erkannt werden. Die L�sung, die Robotinos kontinuierlichen rotieren zu lassen f�hrte in diesem Fall zwar zum Erfolg, beeintr�chtigte aber die Qualit�t des Ausweichman�vers.

Au�erdem kommt es zwischenzeitlich noch immer zu Kollisionen, wenn zu viele Robotios auf einmal aufeinander treffen und der Platz zum Ausweichen zu gering ist. Das Problem ist hier, dass die Robotinos anfangen ein wenig aufzuschwingen, da sie immer wieder in die entgegengesetzte Richtung ausweichen m�ssen. L�sen k�nnte man dieses, in dem die Robotinos langsamer werden, sobald mehrere aufeinander treffen. Alternativ k�nnten weitere Fahrrinnen implementiert werden, so dass es, gerade am Randbereich zwischen Transport- und Fertigungsbereich, nicht zur besagten Situation kommt. 

Zudem k�nnte das Gesamtsystem noch effizienter werden, wenn die Robotinos zwischen zwei gegen�berliegenden Stationen wechseln k�nnten, ohne den Fertigungsbereich verlassen zu m�ssen. Das w�rde daf�r sorgen, dass das Verkehrsaufkommen im Transportbereich gemindert wird. Es w�re energetisch und zeitlich effizienter. 

