\newpage
\chapter{Einf�hrung}

Bei dem Stichwort \glqq Autonome Systeme\grqq{} f�llt der Gedanke schnell auf Industrie 4.0. Die vierte industrielle Revolution, wie sie von dem Bundesministerium f�r Wirtschaft und Energie genannt wird, beschreibt die selbstorganisierte Produktion durch intelligente und digitale Systeme. \cite{bmwi} Ein solches autonomes System soll als Produktionsstra�e im Verbundprojekt der Hochschule f�r Angewandte Wissenschaft Hamburg mit Hilfe von Transportrobotern, sogenannten Robotions, realisiert werden.

Zur Realisierung dieser Produktionsstra�e werden bereits zu Beginn der Projektarbeit alle notwendigen Hardwarekomponenten zur Verf�gung gestellt. Zu diesen Hardwarekomponenten geh�ren die Robotinos. Ein Robotino ist ein mobiles Robotersystem mit omnidirektionalem Antrieb, der es dem Roboter erm�glicht zu jeder Zeit in jede beliebige Richtung fahren zu k�nnen. Sie werden mittels ArUco-Marker und Deckenkameras lokalisiert. Bei den zu transportierenden Werkst�cken handelt es sich um runde Bausteine. Diese Bausteine sind mit einem RFID-Transponder ausgestattet und k�nnen von den Leseger�ten an den Stationen gelesen werden. Insgesamt gibt es vier Stationen, die beidseitig angefahren werden k�nnen. Diese Stationen repr�sentieren die Lager bzw. Maschinen, zu denen die Werkst�cke, je nach Auftrag, transportiert werden m�ssen. Zwei zus�tzliche Stationen sind als Ladestationen f�r die Robotinos ausgelegt.

Auf Grund der Komplexit�t dieses Projektes, wird das Gesamtprojekt in einzelne Aufgabenpakete unterteilt, welche in Gruppenarbeit von drei bis vier Personen zu bearbeiten sind. Insgesamt gibt es f�nf Gewerke, darunter die Auftragskoordination, Bahnplanung  und Regelung, deren Ziel es ist ein funktionsf�higes und zuverl�ssiges Gesamtsystem zu entwickeln. Zus�tzlich wurde sich zum Ziel gesetzt, eine neue Version des Robotinos, den Robotino 2.0, am Ablauf der Produktionsstra�e zu beteiligen.

Um dieses Gesamtziel zu erreichen sind gemeinsame Schnittstellen, stetige Kommunikation, sowie abgestimmtes Zeitmanagement von gro�er Bedeutung. 

In der vorliegenden Dokumentation wird die Umsetzung der Bahnplanung eingehend erl�utert. Zu den Aufgabenbereichen der Bahnplanung geh�rt die Vorgabe eines Weges f�r den Robotino von einem Start- zu einem Zielpunkt, sowie die Kollisionsvermeidung mit statischen und dynamischen Hindernissen. 