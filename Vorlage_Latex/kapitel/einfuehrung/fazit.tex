\chapter{Fazit}
\label{cha:fazit}
Zusammenfassend l�sst sich sagen, dass das ausgew�hlte Konzept, der Potentialfeldmethode, ein sehr sicheres und dynamisches System darstellt. Im Vergleich mit den am Anfang vorgestellten Algorithmen der Bahnplanung besteht dieser Algorithmus aus einem sehr schlankem Grundger�st. Es wird nicht besonders viel Rechenleistung ben�tigt.
Die Erweiterung des Grundsystems um die einzelnen Kollisionsvermeidungsfeatures ist problemlos m�glich.

Die gew�hlte Entwicklungsmethode, die ver�nderten oder neu hinzugef�gten Features direkt am Robotino zu testen, hat die schnelle Entwicklung des Gesamtsystems beg�nstigt.

Das Verbundprojekt hat f�r uns einen sehr gro�en Lernerfolg mit sich gebracht. Vor allem unsere Software-Skills wurden stark verbessert. Der intensive Umgang mit Matlab/Simulink und damit verbundene Stateflowdiagramme hat unsere Programmierkompetenz stark positiv beeinflusst. Was jedoch unserer Meinung nach v�llig im Vordergrund des Verbundprojektes stand, ist die Kommunikation und Zusammenarbeit der verschiedenen Gruppen.
Dieses gesamte Projekt war etwas v�llig Neues f�r uns Studenten. Von Anfang an stand das Gelingen im Gesamtprojekt im Vordergrund. Die f�nf einzelnen Gruppen haben versucht immer in stetiger Kommunikation zu stehen. 
An dieser Stelle w�rden wir gerne noch einmal die wirklich gute Zusammenarbeit mit den Gruppen 3 und 4 hervorheben. Es hat einfach gepasst.

In diesem Projekt sind leider auch negative Seiten einer gro�en Gruppenarbeit aufgetreten. Die fr�h aufgestellte Zeitplanung aller Gruppen wurde nicht von jeder Gruppe eingehalten und es kam fast zu einer Situation in der wir nicht h�tten vortragen k�nnen. Dank intensiver Arbeit von Gruppe 3 und uns kurz vor dem Abschlusstermin konnte der Auftragsdummy noch fertiggestellt werden und eine Pr�sentation war m�glich.

Auch die Betreuung des zust�ndigen Professors war mehr als zufriedenstellend. Gerade in der Anfangsphase, als es um die Auswahl eines geeigneten Konzeptes ging, aber auch w�hrend der gesamten Umsetzung, stand der Professor unterst�tzend zur Seite.

Abschlie�end l�sst sich sagen, dass wir drei viel Spa� an der Projektarbeit hatten und auch aus den negativen Erfahrungen positive Aspekte ziehen konnten.