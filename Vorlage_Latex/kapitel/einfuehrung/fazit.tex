\chapter{Fazit}
\label{cha:fazit}
Zusammenfassend lässt sich sagen dass das ausgewählte Konzept, der Potentialfeldmethode, ein sehr sicheres und dynamisches System darstellt. Im Vergleich mit den am Anfang vorgestellten Algorithmen der Bahnplanung besteht dieser Algorithmus aus einem sehr schlankem Grundgerüst. Es wird nicht besonders viel Rechenleistung benötigt.\\
Die Erweiterung des Grundsystems um die einzelnen Kollisionsvermeidungsfeatures ist problemlos möglich.\\
\\
Die gewählte Entwicklungsmethode die veränderten oder neu hinzugefügten Features direkt am Robotino testen zu können hat die schnelle Entwicklung des Gesamtsystems begünstigt.\\
\\
Das Verbundprojekt hat für uns einen sehr großen Lernerfolg mit sich gebracht. Vor allem unsere Software Skills wurden stark verbessert. Der intensive Umgang mit Matlab/Simulink und damit verbundene Stateflowdiagramme hat unsere Programmierkompetenz stark positiv beeinflusst. Was jedoch unserer Meinung nach völlig im Vordergrund des Verbundprojektes stand ist die Kommunikation und Zusammenarbeit der verschiedenen Gruppen.
Dieses gesamte Projekt war etwas völlig neues für uns Studenten. Von Anfang an stand das Gelingen im Gesamtprojekt im Vordergrund. Die fünf einzelnen Gruppen haben versucht immer in stetiger Kommunikation zu stehen. \\
An dieser Stelle würden wir gerne noch einmal die wirklich gute Zusammenarbeit mit den Gruppen 3 und 4 hervorheben. Es hat einfach gepasst.\\
\\
In diesem Projekt sind leider auch negative Seiten  einer großen Gruppenarbeit aufgetreten. Die früh aufgestellte Zeitplanung aller Gruppen wurde nicht von jeder Gruppe eingehalten und es kam fast zu einer Situation in der wir nicht hätten vortragen können. Dank intensiver Arbeit von Gruppe 3 und uns kurz vor dem Abschlusstermin konnte der Auftragsdummy noch fertiggestellt werden und eine Präsentation war möglich.\\
\\
Abschließend lässt sich sagen dass wir drei viel Spaß an der Projektarbeit hatten und auch aus den negativen Erfahrungen positive Aspekte ziehen konnten.