\newpage

\chapter{Aufgabenstellung}

Aufgabe der Bahnplanung ist es, den Robotino von einer beliebigen Startposition im bekannten Raum zu einem vorgegebenen Ziel fahren zu lassen. Dabei ist zu beachten, dass es zu keiner Zeit zu einer Kollision mit dynamischen oder statischen Hindernissen kommt. 

F�r die Aufnahme und Abgabe der Werkst�cke, also das Werkst�ckhandling allgemein, muss sowohl zu der Auftragskoordination, wie auch zu der Regelung eine geeignete und zuverl�ssige Schnittstelle generiert werden. 
Dazu geh�rt auch das Anfahren an die Stationen und das Fehler-Handling. 

Die Algorithmen zur Realisierung der Bahnplanung und Kollisionsvermeidung sind in Matlab/Simulink zu implementieren. Der interne Programmablauf ist als Simulink/Stateflow mittels Blockschaltbilder zu realisieren.
Die zu erstellende Software wird auf jeden einzelnen Robotino geladen und l�uft dort dezentral als xPC-Target.