\newpage

\chapter{Validierung}
Nachdem die Konzepterstellung abgeschlossen war, wurde mit der Implementierung der Software begonnen. Um bereits ganz zu Beginn die Software testen zu k�nnen, wurde ein Simulationsprogramm erstellt, auf dem das Potentialfeld abgebildet war und die Fahrten der Robotinos nachvollzogen werden konnten. Da in dieser Simulation das Ausweichen zweier Robotinos, wenn diese sich begegnen, nicht eindeutig zu erkennen war, wurde fr�hzeitig begonnen die einzelnen Softwareteile auf die Hardware der Robotinos zu �bertragen und somit am realen System zu testen. 

�ber eine eigens programmierte Software als Dummy k�nnten �ber UDP-Kommunikation manuell Auftr�ge an die Robotinos gesendet werden. In Kooperation mit dem Gewerk3 konnten so vereinzelt Auftr�ge abgearbeitet werden. In diesen Testphasen wurden kleinere Fehler und Ungenauigkeiten erkannt und behoben. So f�hrte beispielsweise das Anstellen in den Fifos zun�chst zu Problemen, da die Potentialfelder zu gro� waren und sich die Robotinnos in Kombination mit den Infrarotsensoren gegenseitig aus der Warteschlange gedr�ngt haben. 
W�hrend der Testphase kam es zu beginn h�ufiger zu Kollisionen mit d�nnen Gegenst�nden, wie beispielsweise einem Stuhlbein. Befinden sich Hindernisse genau zwischen zwei Sensoren, so werden diese nicht erkannt. Gel�st wurde diesen Problem zun�chst mit einer konstanten Rotation der einzelnen Robotinos. 
Wie bereits beschrieben k�nnen die Robotinos sich erkennen und einander ausweichen. Ziel war es einen effizienten Produktionsablauf zu erschaffen. Durch die konstante Rotation der Robotinos verringert sich die Zuverl�ssigkeit des Ausweichman�vers. Nach ausgiebigem Testen wurde die Rotation wieder ein gestellt. Die Priorit�t des Ausweichman�vers wird h�her eingestuft, als die Erkennung schmaler Hindernisse. 

Das Testen mit allen f�nf Robotinos zur gleichen Zeit, war trotz festgelegtem Termin f�r einen Gesamtsystemtest zun�chst nicht m�glich. �ber ein Softwareprogramm, das zuf�llig und kontinuierlich Auftr�ge generiert, k�nnte schlie�lich ein Langzeittest durchgef�hrt werden. Bei den Langzeittest kam es in Situationen, in denen mehrere Robotinos aufeinander treffen und sehr wenig Platz zum Ausweichen war, zu vereinzelt Kollisionen. Diese Kollisionen traten vor allem dann auf, wenn sich in dem Konfigurationsraum alle f�nf Robotinos bewegten. Die noch vorhandene Ungenauigkeit des Robotino 2.0, die besonders durch seine gr��eren Ma�e verursacht werden, und der begrenzte Konfigurationsraum erschweren einen komplett fehlerfreien Ablauf. Zudem kommt es in besagter Situation zum Aufschwingen der Robotinos durch Ausweichen, so dass diese Kollisionen verursacht werden. 

Durch die fr�hzeitigen Test sowohl mit der Simulationssoftware als auch am realen System, wurden Fehler schnell und fr�hzeitig erkannt. Insgesamt l�sst sich sagen, dass ein zuverl�ssiges und robustes System geschaffen wurde, das die Anspr�che an ein funktionsf�higen, autonomen Produktionsablauf gerecht wird. Grunds�tzlich funktioniert das Ausweichen statischer und dynamischer Hindernisse, sei es Mensch oder Robotino, sowie das Anfahren der Stationen und Ladestationen. 

Abschlie�end kann gesagt werden, dass das Ziel, ein System mit f�nf Robotinos zu schaffen, die sich in Ihrem Ablauf selbst organisieren, die �ber die Ist-Zust�nde anderer Systemteilnehmer entscheiden, wo sich ihr eigentliches Ziel befindet, ohne zu wissen, wie der Weg und das Ziel der anderen Robotinos sein wird, wurde zufriedenstellend umgesetzt und ist funktionst�chtig. 
 
