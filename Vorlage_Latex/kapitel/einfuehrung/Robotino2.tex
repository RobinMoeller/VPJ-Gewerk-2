\chapter{Robotino 2.0}\label{cha:Robo2.0}
In diesem Kapitel wird beschrieben, welche Anpassungen am Programm vorgenommen werden m�ssen, um das Programm von den alten Robotinos auf dem neuen Robotino lauff�hig zu bekommen. Um das Programm auf den Robotino 2.0 anzupassen, wird eng mit der Bahnregelung und Gewerk4 zusammengearbeitet. In Absprache mit der Bahnregelung wird eine Anpassung der Ladestationsanfahrpositionen vorgenommen, da der Robotino 2.0 die Ladestation anders anfahren muss als die alten Robotinos. Da f�r den Robotino 2.0 ein anderes Starterkit ben�tigt wird, wird Gewerk�bergreifend ein Robotino 2.0 Starterkit erstellt. Da diese �nderungen am Starterkit im Bahnplanungsblock nur Einfluss auf die UDP Kommunikation mit der Auftragssteuerung hat, wird der Simulinkblock zur UDP-Kommunikation durch ein Output und Input in den gesamt Bahnplanungsblock ersetzen. Da die UDP-Kommunikationsblock von Simulink nicht in Twincat ausgef�hrt werden kann, m�ssen die per Input und Ouput definierten UDP Daten auf den in Twincat integrierten UDP-Kontroller gemappt werden. Da auf dem Robotino 2.0 zu Programmstart die Zielposition und die Robotinoposition auf [0 0 0] gesetzt ist, muss im Programm eine Division durch 0 bei Programmstart vermieden werden, da im Zielpotentialfeld in diesem Fall durch 0 Dividiert wird. Dieses Problem wird durch eine if Abfrage behoben, indem der Zielvektor auf 0 gesetzt wird, wenn die Robotinoposition gleich der Zielposition ist.